\documentclass[a4paper]{article}

\usepackage[english]{babel}
\usepackage[utf8]{inputenc}
\usepackage{amsmath}
\usepackage{graphicx}\\
\graphicspath{{img/}}
\usepackage[colorinlistoftodos]{todonotes}
\usepackage{epigraph}

\title{Numinum SA}

\author{Gaizka Virumbrales \& Rubén Sánchez}

\date{\today}

\begin{document}
\maketitle
\begin{abstract}
La empresa Numinum quiere implementar un SGC y ha pedido al departamento de administración, junto al de TI un plan de implementación para este tipo de sistema.
\end{abstract}


\renewcommand{\contentsname}{Tabla de contenido}
\tableofcontents
\newpage

\section{Visión}
\label{sec:vision}

Ser la empresa reconocida como líder en la venta y producción de electrodomésticos por parte de sus clientes y empleados siguiendo una filosofía de respeto hacia el medio ambiente y de compromiso social.

\section{Misión}
\label{sec:mision}

Contribuir al consumo de energías renovables fabricando electrodomésticos de calidad, duraderos y respetuosos con el medio ambiente que cumplan los estándares de eficiencia energética y contribuir también la igualdad dejando de lado los estándares de la sociedad y aspirando a conseguir unas técnicas de contratación basadas en factores externos al sexo, raza, religión, nacionalidad, etc. Por último, gestionar los datos de forma que creen valor para la compañía como para sus empleados.

\section{Matriz DAFO}

\begin{table}[ht!]
	\centering
	\begin{tabular}{p{6cm}|p{6cm}}
		\textbf{Fortalezas} & \textbf{Debilidades}\\
		- Servir a diferentes estamentos de la sociedad & - No contamos el dinero suficiente para crear y mantener fábricas respetuosas \\
		- Gran popularidad a nivel nacional & - El alto coste de fabricación genera pocos beneficios \\
		- Siete nuevos productos innovadores & \\
		- Nueve calificaciones ISO & \\
		- Sistemas de toma decisiones & \\
		- Wikis y blogs de GC & \\\hline
		\textbf{Oportunidades} & \textbf{Amenazas} \\
		- Subvención por respeto al Medio Ambiente & - La tendencia a la baja de los beneficios podría llevar al cierre de una sucursal o de toda la empresa\\
		- Subvención de colaboración con universidades & \\
		- Investigación de 'la casa inteligente' & \\
		- Capacidad de dar empleo a nuevos investigadores a bajo coste gracias a los estudiantes universitarios
	\end{tabular}
\end{table}

\section{Iniciativas de GC}

\subsection{Páginas amarillas}

\textit{Elaborar unas páginas amarillas generales con el conocimiento tácito de los empleados de cada departamento y las diferentes sedes.}

\begin{figure}[ht!]
	\centering
	\caption{\label{fig:frog}This frog was uploaded to writeLaTeX via the project menu.}{}
\end{figure}

\subsection{Hall Effect}
Explain the classical Hall effect in your own words. What do I measure at $B=0$? And what happens if $B>0$? Which effect gives rise to the voltage drop in the vertical direction?

\subsection{Quantum Hall Effect}
Explain the IQHE in your own words. What does the density of states look like in a 2-DEG when $B=0$? What are Landau levels and how do they arise? What are edge states? What does the electron transport look like when you change the magnetic field? What do you expect to measure?

\section{Experiment 1-2 pages}
\subsection{Fabrication}
Explain a step-by-step recipe for fabrication here. How long did you etch and why? What is an Ohmic contact?
\subsection{Experimental set-up}
Explain the experimental set-up here. Use a schematic picture (make it yourself in photoshop, paint, ...) to show how the components are connected. Briefly explain how a lock-in amplifier works.

\section{Results and interpretation 2-3 pages}
Show a graph of the longitudinal resistivity ($\rho_{xx}$) and Hall resistivity ($\rho_{xy}$) versus magnetic field, extracted from the raw data shown in figure \ref{fig:data}. You will have the link to the data in your absalon messages, if not e-mail Guen (guen@nbi.dk). Explain how you calculated these values, and refer to the theory.

\subsection{Classical regime}
Calculate the sheet electron density $n_{s}$ and electron mobility $\mu$ from the data in the low-field regime, and refer to the theory in section \ref{sec:theory}. Explain how you retrieved the values from the data (did you use a linear fit?).
Round values off to 1 or 2 significant digits: 8.1643 ~= 8.2. Also, 5e-6 is easier to read than 0.000005.

!OBS: This part is optional (only if you have time left).
Calculate the uncertainty as follows: \newline $u(f(x, y, z)) = \sqrt{(\frac{\delta f}{\delta{x}} u(x))^{2} + (\frac{\delta f}{\delta{y}} u(y))^{2} + (\frac{\delta f}{\delta{z}} u(z))^{2}}$, where $f$ is the calculated value ($n_{s}$ or $\mu$), $x, y, z$ are the variables taken from the measurement and $u(x)$ is the uncertainty in x (and so on).

\subsection{Quantum regime}
Calculate $n_{s}$ for the high-field regime.
Show a graph of the longitudinal conductivity ($\rho_{xx}$) and Hall conductivity($\rho_{xy}$) \textbf{in units of the resistance quantum} ($\frac{h}{e^{2}}$), depicting the integer filling factors for each plateau.
Show a graph of the plateau number versus its corresponding value of $1/B$. From this you can determine the slope, which you use to calculate the electron density.
Again, calculate the uncertainty for your obtained values.

\section{Discussion 1/2-1 page}
Discuss your results. Compare the two values of $n_{s}$ that you've found in the previous section. Compare your results with literature and comment on the difference. If you didn't know the value of the resistance quantum, would you be able to deduce it from your measurements? If yes/no, why?

\newpage
\section{Some LaTeX tips}
\label{sec:latex}
\subsection{How to Include Figures}

First you have to upload the image file (JPEG, PNG or PDF) from your computer to writeLaTeX using the upload link the project menu. Then use the includegraphics command to include it in your document. Use the figure environment and the caption command to add a number and a caption to your figure. See the code for Figure \ref{fig:frog} in this section for an example.



\subsection{How to Make Tables}

Use the table and tabular commands for basic tables --- see Table~\ref{tab:widgets}, for example.


\subsection{How to Make Sections and Subsections}

Use section and subsection commands to organize your document. \LaTeX{} handles all the formatting and numbering automatically. Use ref and label commands for cross-references.

\subsection{How to Make Lists}

You can make lists with automatic numbering \dots

\begin{enumerate}
\item Like this,
\item and like this.
\end{enumerate}
\dots or bullet points \dots
\begin{itemize}
\item Like this,
\item and like this.
\end{itemize}
\dots or with words and descriptions \dots
\begin{description}
\item[Word] Definition
\item[Concept] Explanation
\item[Idea] Text
\end{description}

We hope you find write\LaTeX\ useful, and please let us know if you have any feedback using the help menu above.

\begin{thebibliography}{9}
\bibitem{nano3}
  K. Grove-Rasmussen og Jesper Nygård,
  \emph{Kvantefænomener i Nanosystemer}.
  Niels Bohr Institute \& Nano-Science Center, Københavns Universitet

\end{thebibliography}
\end{document}