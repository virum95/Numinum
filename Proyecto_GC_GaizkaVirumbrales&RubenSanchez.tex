\documentclass[a4paper]{article}

\usepackage[english]{babel}
\usepackage[utf8]{inputenc}
\usepackage{amsmath}
\usepackage{graphicx}
\graphicspath{ {img/} }
\usepackage[colorinlistoftodos]{todonotes}
\usepackage{epigraph}

\title{Numinum SA}

\author{Gaizka Virumbrales \& Rubén Sánchez}

\date{\today}

\begin{document}
\maketitle
\begin{abstract}
La empresa Numinum quiere implementar un SGC y ha pedido al departamento de administración, junto al de TI un plan de implementación para este tipo de sistema.
\end{abstract}


\renewcommand{\contentsname}{Tabla de contenido}
\tableofcontents
\newpage

\section{Visión}
\label{sec:vision}

Ser la empresa reconocida como líder en la venta y producción de electrodomésticos por parte de sus clientes y empleados siguiendo una filosofía de respeto hacia el medio ambiente y de compromiso social.

\section{Misión}
\label{sec:mision}

Contribuir al consumo de energías renovables fabricando electrodomésticos de calidad, duraderos y respetuosos con el medio ambiente que cumplan los estándares de eficiencia energética y contribuir también la igualdad dejando de lado los estándares de la sociedad y aspirando a conseguir unas técnicas de contratación basadas en factores externos al sexo, raza, religión, nacionalidad, etc. Por último, gestionar los datos de forma que creen valor para la compañía como para sus empleados.

\section{Matriz DAFO}

\begin{table}[ht!]
	\centering
	\begin{tabular}{p{6cm}|p{6cm}}
		\textbf{Fortalezas} & \textbf{Debilidades}\\
		- Servir a diferentes estamentos de la sociedad & - No contamos el dinero suficiente para crear y mantener fábricas respetuosas \\
		- Gran popularidad a nivel nacional & - El alto coste de fabricación genera pocos beneficios \\
		- Siete nuevos productos innovadores & \\
		- Nueve calificaciones ISO & \\
		- Sistemas de toma decisiones & \\
		- Wikis y blogs de GC & \\\hline
		\textbf{Oportunidades} & \textbf{Amenazas} \\
		- Subvención por respeto al Medio Ambiente & - La tendencia a la baja de los beneficios podría llevar al cierre de una sucursal o de toda la empresa\\
		- Subvención de colaboración con universidades & \\
		- Investigación de 'la casa inteligente' & \\
		- Capacidad de dar empleo a nuevos investigadores a bajo coste gracias a los estudiantes universitarios
	\end{tabular}
\end{table}

\newpage

\section{Iniciativas de GC}

\subsection{Páginas amarillas}

\textit{Elaborar unas páginas amarillas generales con el conocimiento tácito de los empleados de cada departamento y las diferentes sedes.}

\begin{figure}[ht!]
	\includegraphics[width=\textwidth]{EntityRelation.png}
	\caption{Entidad relación de la BD de las páginas amarillas.}
	\label{fig:boat1}
\end{figure}


\subsection{Entrevistas a personas}
\textit{Elegir un encargado por cada país para que actualice una DB con las normas referentes al medio ambiente en ese país, el proceso seguido para adaptarse a esa norma y que haga llegar los cambios al resto de la empresa.}

\subsubsection{Entrevista semiestructurada}
\textit{Tema: Gestión de la normativa de una planta}\\\\
Hola [\textit{Nombre}], veo que actualmente trabajas en el dpto. de [\textit{dpto}], ¿está usted contento con su trabajo?

\begin{itemize}
	\item ¿Estaría dispuesto a asumir responsabilidades extra?
	\item ¿Se considera una persona consciente de lo que pasa a su alrededor?
	\item ¿Sería capaz de contradecir a un cargo superior basándose en una norma establecida?
	\item[\textbf{Entrega}] Le da un texto (un decreto) y tiene que extraer las normas y nuevas leyes que están implícitas en el.
	\item He visto en su CV que tiene conocimientos de SQL, nosotros trabajamos con PLSQL, ¿cree que sería capaz de aprender el lenguaje?
	\item ¿Conoce los procesos básicos de la fábrica?
	\item ¿Cree que sería capaz de modificarlos eficientemente según las normas de medio ambiente?
	\item ¿Tienes familia?
	\begin{itemize}		
	\item[\textbf{Si}] ¿Crees que la familia y el trabajo son aspectos separados de tu vida?
	\item ¿En caso de tener que ir a trabajar a una planta al extranjero, te llevarias a tu familia?
	\item[\textbf{No}] ¿Tienes pensado en tener familia a corto/medio plazo?
	\item ¿Estarias dispuesto a llevar un seguimiento de los cambios en las leyes para poder actualizar las normas de la empresa?
	\end{itemize}
\end{itemize}
Un placer haber tenido esta entrevista con ud ordenaremos las ideas que hemos extraído en un documento y se las enviaremos para verificar que hemos entendido bien todos los puntos. El proceso de selección todavía no ha terminado, en cuanto llegue a su fin le llamaremos con el resultado.

\subsubsection{Entrevista estructurada}
\textit{Tema: Gestión de la normativa de una planta}\\\\
Hola [\textit{Nombre}], es un placer tenerle aquí otra vez

\end{document}